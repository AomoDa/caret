% \VignetteIndexEntry{caret Manual -- Model Building}
% \VignetteDepends{caret}
% \VignettePackage{caret}
\documentclass[12pt]{article}
\usepackage{amsmath}
\usepackage[pdftex]{graphicx}
\usepackage{color}
\usepackage{xspace}
\usepackage{fancyvrb}
\usepackage{fancyhdr}
\usepackage{lastpage}
\usepackage{longtable} 
\usepackage{algorithm2e}
\usepackage[
         colorlinks=true,
         linkcolor=blue,
         citecolor=blue,
         urlcolor=blue]
         {hyperref}
         \usepackage{lscape}

\usepackage{Sweave}

%%%%%%%%%%%%%%%%%%%%%%%%%%%%%%%%%%%%%%%%%%%%%%%%%%%%%%%%%%%%%%%%%%

% define new colors for use
\definecolor{darkgreen}{rgb}{0,0.6,0}
\definecolor{darkred}{rgb}{0.6,0.0,0}
\definecolor{lightbrown}{rgb}{1,0.9,0.8}
\definecolor{brown}{rgb}{0.6,0.3,0.3}
\definecolor{darkblue}{rgb}{0,0,0.8}
\definecolor{darkmagenta}{rgb}{0.5,0,0.5}

%%%%%%%%%%%%%%%%%%%%%%%%%%%%%%%%%%%%%%%%%%%%%%%%%%%%%%%%%%%%%%%%%%

\newcommand{\bld}[1]{\mbox{\boldmath $#1$}}
\newcommand{\shell}[1]{\mbox{$#1$}}
\renewcommand{\vec}[1]{\mbox{\bf {#1}}}

\newcommand{\ReallySmallSpacing}{\renewcommand{\baselinestretch}{.6}\Large\normalsize}
\newcommand{\SmallSpacing}{\renewcommand{\baselinestretch}{1.1}\Large\normalsize}

\newcommand{\halfs}{\frac{1}{2}}

\setlength{\oddsidemargin}{-.25 truein}
\setlength{\evensidemargin}{0truein}
\setlength{\topmargin}{-0.2truein}
\setlength{\textwidth}{7 truein}
\setlength{\textheight}{8.5 truein}
\setlength{\parindent}{0.20truein}
\setlength{\parskip}{0.10truein}

%%%%%%%%%%%%%%%%%%%%%%%%%%%%%%%%%%%%%%%%%%%%%%%%%%%%%%%%%%%%%%%%%%
\pagestyle{fancy}
\lhead{}
\chead{The {\tt caret} Package}
\rhead{}
\lfoot{}
\cfoot{}
\rfoot{\thepage\ of \pageref{LastPage}}
\renewcommand{\headrulewidth}{1pt}
\renewcommand{\footrulewidth}{1pt}
%%%%%%%%%%%%%%%%%%%%%%%%%%%%%%%%%%%%%%%%%%%%%%%%%%%%%%%%%%%%%%%%%%

\title{The {\tt caret} Package}
\author{Max Kuhn \\ max.kuhn@pfizer.com}


\begin{document}

\maketitle

\thispagestyle{empty}
	

\section{Model Training and Parameter Tuning}\label{S:train}

\texttt{caret} has several functions that attempt to streamline the model building and evaluation process. 

The \texttt{train} function can be used to
\begin{itemize}
   \item evaluate, using resampling, the effect of model tuning parameters on performance
   \item choose the ``optimal'' model across these parameters 
   \item estimate model performance from a training set
\end{itemize}

To optimize tuning parameters of models, \texttt{train} can be used to fit many predictive models over a grid of parameters and return the ``best'' model (based on resampling statistics). See Table \ref{T:methods} for the models currently available.

As an example, the multidrug resistance reversal (MDRR) agent data is used to determine a predictive model for the ``ability of a compound to reverse a leukemia cell's resistance to adriamycin'' (\href{http://pubs.acs.org/cgi-bin/abstract.cgi/jcisd8/2005/45/i03/abs/ci0500379.html}{Svetnik et al, 2003}). For each sample (i.e. compound), predictors are calculated that reflect characteristics of the molecular structure. These molecular descriptors are then used to predict assay results that reflect resistance. 

The data are accessed using \texttt{data(mdrr)}. This creates a data frame of predictors called \texttt{mdrrDescr} and a factor vector with the observed class called \texttt{mdrrClass}.

To start, we will:
 
\begin{itemize}
   \item use unsupervised filters to remove predictors with unattractive characteristics (e.g. spare distributions or high inter--predictor correlations)
   \item split the entire data set into a training and test set
   \item center and scale the training and test set using the predictor means and standard deviations from the training set
\end{itemize}

See the package vignette ``caret Manual -- Data and Functions'' for more details about these operations.

\begin{small}
\begin{Schunk}
\begin{Sinput}
> print(ncol(mdrrDescr))
\end{Sinput}
\begin{Soutput}
[1] 342
\end{Soutput}
\begin{Sinput}
> nzv <- nearZeroVar(mdrrDescr)
> filteredDescr <- mdrrDescr[, -nzv]
> print(ncol(filteredDescr))
\end{Sinput}
\begin{Soutput}
[1] 297
\end{Soutput}
\begin{Sinput}
> descrCor <- cor(filteredDescr)
> highlyCorDescr <- findCorrelation(descrCor, cutoff = 0.75)
> filteredDescr <- filteredDescr[, -highlyCorDescr]
> print(ncol(filteredDescr))
\end{Sinput}
\begin{Soutput}
[1] 50
\end{Soutput}
\begin{Sinput}
> set.seed(1)
> inTrain <- sample(seq(along = mdrrClass), length(mdrrClass)/2)
> trainDescr <- filteredDescr[inTrain, ]
> testDescr <- filteredDescr[-inTrain, ]
> trainMDRR <- mdrrClass[inTrain]
> testMDRR <- mdrrClass[-inTrain]
> print(length(trainMDRR))
\end{Sinput}
\begin{Soutput}
[1] 264
\end{Soutput}
\begin{Sinput}
> print(length(testMDRR))
\end{Sinput}
\begin{Soutput}
[1] 264
\end{Soutput}
\begin{Sinput}
> preProcValues <- preProcess(trainDescr)
> trainDescr <- predict(preProcValues, trainDescr)
> testDescr <- predict(preProcValues, testDescr)
\end{Sinput}
\end{Schunk}
\end{small}

To estimate model performance across the tuning parameters ``leave group out cross--validation'' (\texttt{LGOCV}) can be used. This technique is repeated splitting of the data into training and test sets (without replacement). If the resampling method is not specified, simple bootstrapping is used. To train a support vector machine classification model (radial basis function kernel) on these multidrug resistance reversal agent data, we can first setup a control object\footnote{This is optional; to use the default specifications, the control object does not need to be specified} that specifies the type of resampling used, the number of data splits (30), the proportion of data in the sub--training sets (75$\%$) and whether the iterations should be printed as they occur. In this case, we need to specify the proportion of samples used in each resampled training set. We also set the seed.

\begin{small}
\begin{Schunk}
\begin{Sinput}
> fitControl <- trainControl(method = "LGOCV", p = 0.75, number = 30, 
+     returnResamp = "all", verboseIter = FALSE)
> set.seed(2)
\end{Sinput}
\end{Schunk}
\end{small}


The first two arguments to \texttt{train} are the predictor and outcome data objects, respectively. The third argument, \texttt{method}, specifies the type of model. For this model, the tuning parameters are the cost value (the \texttt{C} argument in \texttt{kernlab}'s \texttt{ksvm} function) and the radius of the RBF (the \texttt{sigma} argument to the kernel function). The \texttt{tuneLength} argument sets the size of the grid used to search the tuning parameter space and \texttt{trControl} is the control parameter for the \texttt{train} function. 

\begin{small}
\begin{Schunk}
\begin{Sinput}
> svmFit <- train(trainDescr, trainMDRR, method = "svmRadial", 
+     tuneLength = 4, trControl = fitControl)